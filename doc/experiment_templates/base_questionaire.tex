\documentclass{scrartcl}
\usepackage{german}
\usepackage[latin1]{inputenc}
\usepackage[german]{babel}
\usepackage{amssymb}
\usepackage{graphicx}
\usepackage{fancyhdr}
\usepackage{lastpage}
\usepackage{tikz}
\usepackage{ifthen}
\setlength{\parskip}{\medskipamount}
\setlength{\parindent}{0pt}


%%%%%%%%%%%%%%%%%%%%%%%%
% Header-/Footer %
%%%%%%%%%%%%%%%%%%%%%%%%
\pagestyle{fancy}
\lhead{VPcode:}
%\chead{VCO - Basisfragebogen}
\chead{Datum: }
\rhead{
  Seite \thepage{} von \pageref{LastPage}
}
\lfoot{}
\cfoot{}
\rfoot{} 

\newcommand{\openquestion}[2]{%
\large{#2}\\
\vspace*{0.2cm}\\
\underline{\hspace{0.8\textwidth}}
\vspace*{0.25cm}\\
}

%-1 is placeholder, values >= 0 are drawn on scale
\newcommand{\scalequestion}[5][{-1}]{%
\large{#3}\\
\vspace*{0.1cm}\\
\begin{tikzpicture}[x=1.4cm]
\draw[-, semithick] (0,0) -- (10,0) ;
\draw[-, semithick] (0, -0.25) -- (0, 0.25);
\draw node[right] at (0, -0.5) {\small #4};
\draw[-, semithick] (10, -0.25) -- (10, 0.25);
\draw node[left] at (10, -0.5) {\small #5};
\ifthenelse{ \NOT #1 < 0}{
%draw a marker at the position
  \draw[-, thick] (#1-0.1, -0.2) -- (#1+0.1, 0.3);
  \draw[-, thick] (#1-0.1, 0.3) -- (#1+0.1, -0.2);
}{}
\end{tikzpicture}
\vspace*{0.25cm}
}

%%%%%%%%%%%%%%%%%%%%%%%%
% Anfang des Dokuments %
%%%%%%%%%%%%%%%%%%%%%%%%
\begin{document}

\section*{Fragebogen}

\subsection*{Beispiel}
\scalequestion[9]{Bsp}{Wie motiviert sind Sie?}{unmotiviert}{hoch motiviert}

\newpage
\subsection*{Autofahren}

\openquestion{1a}{Seit wie vielen Jahren besitzen Sie einen F�hrerschein?}

\openquestion{2a}{Wie viele Jahreskilometer legen Sie gesch�tzt als
  Autofahrerin bzw. Autofahrer zur�ck?}

\scalequestion{3a}{Als wie erfahren w�rden Sie sich als Fahrerin
  bzw. Fahrer bezeichnen?}{unerfahren}{erfahren}


\scalequestion{4a}{Als wie defensiv w�rden Sie sich als Fahrerin
  bzw. Fahrer bezeichnen?}{defensiv}{sportlich}



\scalequestion{5a}{Wie nehmen Sie Ihre Umgebung beim Autofahren
  wahr?}{bewusst}{unbewusst}


\scalequestion{6a}{Nehmen Sie h�ufig m�gliche Unfallgefahren beim
  Autofahren wahr?}{h�ufig}{selten}

\scalequestion{7a}{Wie hoch w�rden Sie Ihre Risikobereitschaft beim
  Autofahren bezeichnen?}{hoch}{niedrig}

\subsection*{Fahrradfahren}

\scalequestion{8a}{Als wie erfahren w�rden Sie sich als Radfahrer
  bezeichnen?}{unerfahren}{erfahren}

\scalequestion{9a}{Als wie defensiv w�rden Sie sich als Radfahrer
  bezeichnen?}{defensiv}{sportlich}

\scalequestion{10a}{Wie nehmen Sie Ihre Umgebung beim Radfahren
  wahr?}{bewusst}{unbewusst}

\scalequestion{11a}{Nehmen Sie h�ufig m�gliche Unfallgefahren beim
  Radfahren wahr?}{h�ufig}{selten}

\scalequestion{12a}{Wie hoch w�rden Sie Ihre Risikobereitschaft beim Radfahren bezeichnen?}{hoch}{niedrig}

\end{document}


